\section{Introduction}

The project undertaken involved solving an Indonesian Dot Puzzle using different state space search algorithms, in python. Amongst the algorithms, there was one uniformed search algorithm (Depth-First or DFS) and two informed search algorithms (Best-First or BFS, and A*). 
\\

Before tackling coding, it was important that an overall architecture was set up, which evolved throughout several stages of the project. In the end, our project was divided into several entities: Nodes, Agents, and Puzzles. The Node can be considered as the basis of a graph tree, which in our case represented a board’s state. Anything relating to the node, including its predecessor, descendent, and characteristics (depth, size, and length) would be located within this module. The Agents module was created specifically for each algorithm. In terms of exploration on a visited node, Best-First and A* both stem from a Breadth-First approach. However, due to the professor’s request of ordering each children of a node in a specific format (boards containing 1’s at the top leftmost spot would be selected first) for DFS, it essentially mirrored the Breadth-First approach. Hence, the Agent’s module was customized to simply calculates a node’s heuristic, specific to each algorithm.  Lastly, the Puzzle module was implemented to handle bulk of the work. Traversal, solution handling and search handling was performed by this module.
\\

For each algorithm, a basic structure was created in python upon which adjustments were performed in order to improve the algorithm’s running efficiency and time. Since our code did not involve any complex computations and state handling was done by converting each state into an integer (considering it originally as a binary), we were able to forgo using Numpy, speeding up our solution time. In the end, our goal was to design a working modular environment in which we would be able to solve an Indonesian Dot puzzle as efficiently as possible for each algorithm.
