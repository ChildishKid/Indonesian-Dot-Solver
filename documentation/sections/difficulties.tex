\section{Challenges}

\subsection{Relational Modelling}

A large portion of time leading to the completion of the project was attributed to understanding how subjects should be linked. This process was judged to be difficult since the domain requirements can be ambiguous at times. To resolve this issue, an experiment was conducted using the three main subjects of the domain: agents (search algorithms), puzzles (graphs), and nodes (board state).
\\
\\

(i) Agents use a puzzles and nodes. 
\\

(ii) Puzzles use agents and nodes.
\\

(iii) Each agent uses puzzles and nodes.
\\

(iv) Each puzzle uses agents and nodes.
\\
\\

When the models were compared on the same data set, it was observed that the average run-time for models (i) to (iv) was 2 minutes, 66 ms, 102 ms, and 18 ms respectively.  Moreover, results of the experiment led to two important findings that helped us classify the domain requirements of the project.\\

The first finding resolves the concern of whether or not to use agent or puzzle as the primary subject of the system. For example, should (i/iii) be chosen over (i/iv)? If agent was the primary subject, then each agent must have a reference to a puzzle, and each agent would rely on the puzzle to provide it with a new state for every step of a traversal. The issue with this is that each agent had similar implementations for traversing the puzzle. This affected the overall performance of our system because there were redundancies for the traversal behaviour. To resolve this issue, we took away the responsibility of traversing nodes from the agent class and assigned it to the puzzle class. As a result of this change, the agent's responsibility was now limited to computing heuristic values. Overall, this finding suggests against using (i) and (iii) as a schema for modelling the system.\\

The first finding also suggests that agents should not have direct access to the node class. Yet, having access to this object is important for the system to function properly since it stores the values of each heuristic. A first approach to this problem was to include an accesssor method in the puzzle class. However, this increased the time complexity for state searching by a linear factor, and also contradicts the schema proposed in (ii) and (iv). Therefore, the puzzle needs to reference the agent.\\

\begin{figure}[H]
\includegraphics[width=0.75\linewidth]{assets/functional.png}
\caption{Sequence Diagram of Indonesian Dot Solver} \label{fig1}
\end{figure}

This leads to the next and last finding: A puzzle and an agent cannot have a many-to-many relational type because agent no longer has a reference to a puzzle. Moreover, many nodes are used by one puzzle. Therefore, the combination of both the agent and the node identifies the puzzle that it belongs to. This makes sense since each node points to it's predecessor, with the root node uniquely linked to the puzzle it belongs to (assuming there are no duplicate lines in the test file). Therefore, the second finding suggests that the logical architecture should be modelled after statement (iv).\\

Overall, by performing this experimentation and discovering why the logical architecture performed well under different scenarios, we were able to create a classification plan for classifying each domain requirement into one that belongs to agent, puzzles, or nodes. This experimentation also helped us maximize the performance of the overall system.

\begin{figure}[H]
\includegraphics[width=0.75\linewidth]{assets/schema.png}
\caption{Entity Relationship Diagram of Indonesian Dot Solver} \label{fig2}
\end{figure}
